\section*{Introduction}
\addcontentsline{toc}{section}{Introduction}
\label{chap:introduction}

The objective of this project is to perform a structural analysis of a motorcycle suspension spring, specifically the rear spring of a Kawasaki ZR-7. This analysis applies the Finite Element Method (FEM) \cite{course_notes} to solve a practical engineering problem.

The project pursues four main goals, as outlined in the assignment description \cite{project_assignment}. First, we aim to understand the mathematical formulation of finite elements, focusing on the linear tetrahedral element (TETRA4). Second, the project requires gaining proficiency in Salome-Meca for pre-processing and post-processing tasks. Third, we must build a custom finite element code in Python that implements stiffness matrix assembly, boundary condition application, and stress computation. Finally, the project involves a critical analysis of the results to understand the limitations of the chosen element types and the assumptions made during modeling.

This report is divided into two parts. Part I describes the model setup for the Kawasaki ZR-7 spring, presents the numerical results compared with experimental data, and discusses the validity of our assumptions. Part II details the verification of our custom Python implementation through patch tests and convergence studies.
