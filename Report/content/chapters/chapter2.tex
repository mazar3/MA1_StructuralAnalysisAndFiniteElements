\section{Numerical results}
\label{chap:results}

This chapter presents the results of the Finite Element analyses. We evaluate the convergence behavior of our models and compare the computed stiffness values with experimental measurements.

\subsection{Plate with hole: stress convergence}
The maximum Von Mises stress at the hole edge was computed for four mesh densities. The results are summarized in Table \ref{tab:plate_results} and visualized in Figure \ref{fig:plate_convergence}.

The theoretical maximum stress, calculated using the analytical solution for an infinite plate with a hole ($K_t \approx 3$), is $\sigma_{max} = 236.56$ MPa. The "Very Fine" mesh yields a value of 233.48 MPa, which is within 1.3\% of the theoretical target, demonstrating excellent convergence.

\subsection{Front spring stiffness}
The stiffness $k$ of the front spring was calculated for the 1-coil, 2-coil, and 3-coil models across four mesh densities. The results were extrapolated to the full spring (26 coils total, 24 active) and are presented in Table \ref{tab:front_spring_results} and Figure \ref{fig:front_spring_convergence}.

Comparing the three geometric models in Table \ref{tab:front_spring_results}, we observe that adding coils progressively reduces the stiffness for a given mesh density (e.g., from 12.20 kN/m to 11.90 kN/m for the Coarse mesh). This confirms that boundary effects are better captured by the 3-coil model.

The experimental stiffness is $k_{exp} = 7.2$ kN/m. The 3-coil model with the "Very Fine" mesh performs best, predicting a stiffness of 7.73 kN/m. This overestimation is typical for TETRA4 elements, which tend to be overly stiff, but the value is reasonably close to reality.

\subsection{Rear spring stiffness}
For the rear spring (6 coils total, 4 active), the numerical results are shown in Table \ref{tab:rear_spring_results} and plotted in Figure \ref{fig:rear_spring_convergence}.

The experimental stiffness is $k_{exp} = 85.3$ kN/m. Interestingly, the "Fine" mesh result (85.16 kN/m) is incredibly close to the experimental value, while the more refined "Very Fine" mesh yields a lower stiffness (78.14 kN/m, $-8.4\%$ error). This counter-intuitive result suggests that the "Fine" mesh result was fortuitously accurate due to error cancellation, whereas the "Very Fine" model has converged to a numerical solution that differs from the experimental reality (likely due to geometric simplifications or boundary condition idealizations).
